% Robert Adams	02/07/2012	CS 311

\documentclass[letterpaper,10pt]{article} %twocolumn titlepage 
\usepackage{graphicx}
\usepackage{amssymb}
\usepackage{amsmath}
\usepackage{amsthm}

\usepackage{alltt}
\usepackage{float}
\usepackage{color}
\usepackage{url}

\usepackage{balance}
\usepackage[TABBOTCAP, tight]{subfigure}
\usepackage{enumitem}
\usepackage{pstricks, pst-node}


\usepackage{geometry}
\geometry{margin=1in, textheight=8.5in} %textwidth=6in

%random comment

\newcommand{\cred}[1]{{\color{red}#1}}
\newcommand{\cblue}[1]{{\color{blue}#1}}

\usepackage{hyperref}

\def\name{Robert M Adams}

%% The following metadata will show up in the PDF properties
\hypersetup{
  colorlinks = true,
  urlcolor = black,
  pdfauthor = {\name},
  pdfkeywords = {cs311 ``operating systems'' sieve primes threaded},
  pdftitle = {CS 311 Project 3: A threaded implementation Sieve of Eratosthenes},
  pdfsubject = {CS 311 Project 3},
  pdfpagemode = UseNone
}


\begin{document}
  \title{CS 311 Project 3: A threaded implementation of Sieve of Eratosthenes}
  \author{Robert Adams}
\maketitle


\section{Design Decisions}


I choose Sieve of Eratosthenes method of finding primes as the large loops it requires to mark numbers as non prime seemed a good candidate for parallel running.
 Although I believe I could malloc a bitmap and pass it into the thread's function, for simplicity I made it global along with the max number and the current number to be checked.
  All threads share and increment the current odd number to be checked as prime. I used macros for the bitmap manipulation to make it easier to read the program.


\section{Difficulties}


Although I was able to come up with a serial implementation of Sieve of Eratosthenes, coming up with the structure to effectively parallelize the algorithm took a god deal of refinement.  
I started with an implementation that found the primes in one thread and marked them in another but I found it difficult to synchronize the data and eventually ended up with something that worked but was effectively serial.  
It used conditions to mutex locks to block everything until a prime was found, then the primes loop would be locked until all non-primes were marked.  
It took a step back and other look a day later to find a solution without locking, which is the only way an implementation could find 2\verb|^|32 primes in under 45 secs. My algorithm was able to find all primes in 2\verb|^|32 in 17 secs with 8 threads on my personal computer.



  \begin{figure}[p]
    \centering
    \input{plot.tex}
    \caption{Runtimes with 8 threads until 2\textasciicircum32. }
    \label{runtimes}
  \end{figure}

  \begin{figure}[p]
    \centering
    % GNUPLOT: LaTeX picture
\setlength{\unitlength}{0.240900pt}
\ifx\plotpoint\undefined\newsavebox{\plotpoint}\fi
\sbox{\plotpoint}{\rule[-0.200pt]{0.400pt}{0.400pt}}%
\begin{picture}(1350,810)(0,0)
\sbox{\plotpoint}{\rule[-0.200pt]{0.400pt}{0.400pt}}%
\put(171.0,131.0){\rule[-0.200pt]{4.818pt}{0.400pt}}
\put(151,131){\makebox(0,0)[r]{ 20}}
\put(1269.0,131.0){\rule[-0.200pt]{4.818pt}{0.400pt}}
\put(171.0,237.0){\rule[-0.200pt]{4.818pt}{0.400pt}}
\put(151,237){\makebox(0,0)[r]{ 40}}
\put(1269.0,237.0){\rule[-0.200pt]{4.818pt}{0.400pt}}
\put(171.0,344.0){\rule[-0.200pt]{4.818pt}{0.400pt}}
\put(151,344){\makebox(0,0)[r]{ 60}}
\put(1269.0,344.0){\rule[-0.200pt]{4.818pt}{0.400pt}}
\put(171.0,450.0){\rule[-0.200pt]{4.818pt}{0.400pt}}
\put(151,450){\makebox(0,0)[r]{ 80}}
\put(1269.0,450.0){\rule[-0.200pt]{4.818pt}{0.400pt}}
\put(171.0,556.0){\rule[-0.200pt]{4.818pt}{0.400pt}}
\put(151,556){\makebox(0,0)[r]{ 100}}
\put(1269.0,556.0){\rule[-0.200pt]{4.818pt}{0.400pt}}
\put(171.0,663.0){\rule[-0.200pt]{4.818pt}{0.400pt}}
\put(151,663){\makebox(0,0)[r]{ 120}}
\put(1269.0,663.0){\rule[-0.200pt]{4.818pt}{0.400pt}}
\put(171.0,769.0){\rule[-0.200pt]{4.818pt}{0.400pt}}
\put(151,769){\makebox(0,0)[r]{ 140}}
\put(1269.0,769.0){\rule[-0.200pt]{4.818pt}{0.400pt}}
\put(171.0,131.0){\rule[-0.200pt]{0.400pt}{4.818pt}}
\put(171,90){\makebox(0,0){ 1}}
\put(171.0,749.0){\rule[-0.200pt]{0.400pt}{4.818pt}}
\put(295.0,131.0){\rule[-0.200pt]{0.400pt}{4.818pt}}
\put(295,90){\makebox(0,0){ 2}}
\put(295.0,749.0){\rule[-0.200pt]{0.400pt}{4.818pt}}
\put(419.0,131.0){\rule[-0.200pt]{0.400pt}{4.818pt}}
\put(419,90){\makebox(0,0){ 3}}
\put(419.0,749.0){\rule[-0.200pt]{0.400pt}{4.818pt}}
\put(544.0,131.0){\rule[-0.200pt]{0.400pt}{4.818pt}}
\put(544,90){\makebox(0,0){ 4}}
\put(544.0,749.0){\rule[-0.200pt]{0.400pt}{4.818pt}}
\put(668.0,131.0){\rule[-0.200pt]{0.400pt}{4.818pt}}
\put(668,90){\makebox(0,0){ 5}}
\put(668.0,749.0){\rule[-0.200pt]{0.400pt}{4.818pt}}
\put(792.0,131.0){\rule[-0.200pt]{0.400pt}{4.818pt}}
\put(792,90){\makebox(0,0){ 6}}
\put(792.0,749.0){\rule[-0.200pt]{0.400pt}{4.818pt}}
\put(916.0,131.0){\rule[-0.200pt]{0.400pt}{4.818pt}}
\put(916,90){\makebox(0,0){ 7}}
\put(916.0,749.0){\rule[-0.200pt]{0.400pt}{4.818pt}}
\put(1041.0,131.0){\rule[-0.200pt]{0.400pt}{4.818pt}}
\put(1041,90){\makebox(0,0){ 8}}
\put(1041.0,749.0){\rule[-0.200pt]{0.400pt}{4.818pt}}
\put(1165.0,131.0){\rule[-0.200pt]{0.400pt}{4.818pt}}
\put(1165,90){\makebox(0,0){ 9}}
\put(1165.0,749.0){\rule[-0.200pt]{0.400pt}{4.818pt}}
\put(1289.0,131.0){\rule[-0.200pt]{0.400pt}{4.818pt}}
\put(1289,90){\makebox(0,0){ 10}}
\put(1289.0,749.0){\rule[-0.200pt]{0.400pt}{4.818pt}}
\put(171.0,131.0){\rule[-0.200pt]{0.400pt}{153.694pt}}
\put(171.0,131.0){\rule[-0.200pt]{269.326pt}{0.400pt}}
\put(1289.0,131.0){\rule[-0.200pt]{0.400pt}{153.694pt}}
\put(171.0,769.0){\rule[-0.200pt]{269.326pt}{0.400pt}}
\put(30,450){\makebox(0,0){sec}}
\put(730,29){\makebox(0,0){number of threads}}
\put(171,721){\usebox{\plotpoint}}
\multiput(171.58,714.00)(0.499,-1.988){245}{\rule{0.120pt}{1.687pt}}
\multiput(170.17,717.50)(124.000,-488.498){2}{\rule{0.400pt}{0.844pt}}
\multiput(295.00,229.58)(1.270,0.498){95}{\rule{1.112pt}{0.120pt}}
\multiput(295.00,228.17)(121.691,49.000){2}{\rule{0.556pt}{0.400pt}}
\multiput(419.00,276.92)(0.638,-0.499){193}{\rule{0.610pt}{0.120pt}}
\multiput(419.00,277.17)(123.733,-98.000){2}{\rule{0.305pt}{0.400pt}}
\multiput(544.00,180.58)(3.505,0.495){33}{\rule{2.856pt}{0.119pt}}
\multiput(544.00,179.17)(118.073,18.000){2}{\rule{1.428pt}{0.400pt}}
\multiput(668.00,196.93)(9.417,-0.485){11}{\rule{7.186pt}{0.117pt}}
\multiput(668.00,197.17)(109.086,-7.000){2}{\rule{3.593pt}{0.400pt}}
\multiput(792.00,189.92)(6.437,-0.491){17}{\rule{5.060pt}{0.118pt}}
\multiput(792.00,190.17)(113.498,-10.000){2}{\rule{2.530pt}{0.400pt}}
\multiput(916.00,181.58)(1.532,0.498){79}{\rule{1.320pt}{0.120pt}}
\multiput(916.00,180.17)(122.261,41.000){2}{\rule{0.660pt}{0.400pt}}
\multiput(1041.00,220.92)(2.729,-0.496){43}{\rule{2.257pt}{0.120pt}}
\multiput(1041.00,221.17)(119.316,-23.000){2}{\rule{1.128pt}{0.400pt}}
\put(1165,197.17){\rule{24.900pt}{0.400pt}}
\multiput(1165.00,198.17)(72.319,-2.000){2}{\rule{12.450pt}{0.400pt}}
\put(171,721){\makebox(0,0){$+$}}
\put(295,229){\makebox(0,0){$+$}}
\put(419,278){\makebox(0,0){$+$}}
\put(544,180){\makebox(0,0){$+$}}
\put(668,198){\makebox(0,0){$+$}}
\put(792,191){\makebox(0,0){$+$}}
\put(916,181){\makebox(0,0){$+$}}
\put(1041,222){\makebox(0,0){$+$}}
\put(1165,199){\makebox(0,0){$+$}}
\put(1289,197){\makebox(0,0){$+$}}
\put(171,721){\usebox{\plotpoint}}
\multiput(171,721)(5.072,-20.126){25}{\usebox{\plotpoint}}
\multiput(295,229)(19.303,7.628){6}{\usebox{\plotpoint}}
\multiput(419,278)(16.334,-12.806){8}{\usebox{\plotpoint}}
\multiput(544,180)(20.540,2.982){6}{\usebox{\plotpoint}}
\multiput(668,198)(20.723,-1.170){6}{\usebox{\plotpoint}}
\multiput(792,191)(20.688,-1.668){6}{\usebox{\plotpoint}}
\multiput(916,181)(19.722,6.469){6}{\usebox{\plotpoint}}
\multiput(1041,222)(20.407,-3.785){6}{\usebox{\plotpoint}}
\multiput(1165,199)(20.753,-0.335){6}{\usebox{\plotpoint}}
\put(1289,197){\usebox{\plotpoint}}
\put(171,721){\makebox(0,0){$\times$}}
\put(295,229){\makebox(0,0){$\times$}}
\put(419,278){\makebox(0,0){$\times$}}
\put(544,180){\makebox(0,0){$\times$}}
\put(668,198){\makebox(0,0){$\times$}}
\put(792,191){\makebox(0,0){$\times$}}
\put(916,181){\makebox(0,0){$\times$}}
\put(1041,222){\makebox(0,0){$\times$}}
\put(1165,199){\makebox(0,0){$\times$}}
\put(1289,197){\makebox(0,0){$\times$}}
\put(171.0,131.0){\rule[-0.200pt]{0.400pt}{153.694pt}}
\put(171.0,131.0){\rule[-0.200pt]{269.326pt}{0.400pt}}
\put(1289.0,131.0){\rule[-0.200pt]{0.400pt}{153.694pt}}
\put(171.0,769.0){\rule[-0.200pt]{269.326pt}{0.400pt}}
\end{picture}

    \caption{Runtimes 1-8 threads at 2\textasciicircum32 upper limit.}
    \label{runtimes}
  \end{figure}

\begin{table}
\centering
    \begin{tabular}{ |l|l| }
\hline
        Date                           & Description                                                                  \\ \hline
 Sun Feb 26 12:01:03 2012 -0800      & changed output for gnuplot    \\
Sun Feb 26 11:21:03 2012 -0800       & comment out set style in gnuplot \\
 Sat Feb 25 19:01:19 2012 -0800     &   minor refactoring   \\
Sat Feb 25 18:30:31 2012 -0800      &   better timing code, will do 2\verb|^|32 2threads 41sec \\
Sat Feb 25 17:46:21 2012 -0800      &   runs to sqrt of n   \\
Sat Feb 25 16:48:15 2012 -0800       &   fixed testnum = 1, runs 429496729 in 61 sec \\
Sat Feb 25 16:12:48 2012 -0800      &   better structure, missing some primes though    \\
Sat Feb 25 12:39:24 2012 -0800      &   added another condition variable, pretty much just works like a serial alg now...   \\
Fri Feb 24 17:41:29 2012 -0800      &   changed limit to max. Working, but only with sleep(1)   \\
Fri Feb 24 16:38:46 2012 -0800      &   hreads start in the correct order   \\
Fri Feb 24 16:27:58 2012 -0800      &   reworked no infinit loop. gets first prime and hangs    \\
Fri Feb 24 15:04:37 2012 -0800      &    in the process of adding pthread cond and mutex    \\
Tue Feb 21 20:31:37 2012 -0800      &   doesnt calc first primes, starts at 17  \\
Sat Feb 18 14:53:21 2012 -0800      &   added example for sieve primes w/ bitmap    \\
Fri Feb 17 20:06:10 2012 -0800      &   added proper include for compile with sleep \\
Fri Feb 17 15:55:06 2012 -0800      &    has pthread tutorial code  \\

\hline
    \end{tabular}
\caption{Pulled from git commit log.}\label{commit-logs}
\end{table}


\end{document}
